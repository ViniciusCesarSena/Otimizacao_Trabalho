\documentclass[12pt,a4paper]{article}
\usepackage[margin=2.5cm]{geometry}
\usepackage{amsmath}
\usepackage{booktabs}
\usepackage{array}
\usepackage{float}

\title{Trabalho de Otimizacao\\Relatorio Tecnico}
\author{Vinicius Cesar Sena Torres (RA 22409225)}
\date{25 de Outubro de 2025}

\begin{document}

\maketitle

\section*{1. Introducao}
Este relatorio descreve a resolucao do problema de levar um robo do ponto inicial $(0,0)$ ao ponto final $(10,10)$ no intervalo continuo $t \in [0,5]$, minimizando a energia acumulada ao longo do percurso. A energia integra o quadrado da velocidade instantanea somado ao termo $\sin(t)$. O problema foi formulado como um Problema de Programacao Nao Linear (NLP), modelado com Pyomo e resolvido com o solver IPOPT.

Os objetivos principais foram:
\begin{itemize}
    \item Formular matematicamente o problema continuo e sua versao discretizada.
    \item Implementar o modelo no Pyomo.
    \item Resolver o problema com IPOPT e analisar os resultados.
    \item Registrar os achados e discutir possiveis extensoes.
\end{itemize}

\section*{2. Formulacao Matematica}
\subsection*{2.1 Problema continuo}
A energia a minimizar e dada por:
\[
E = \int_{0}^{5} \left( v(t)^2 + \sin(t) \right) \, dt,
\]
onde $v(t) = \sqrt{\dot{x}(t)^2 + \dot{y}(t)^2}$.

As restricoes de contorno sao:
\[
x(0) = 0, \quad y(0) = 0, \qquad x(5) = 10, \quad y(5) = 10.
\]

\subsection*{2.2 Discretizacao}
O intervalo $[0,5]$ foi dividido em $N = 5$ subintervalos uniformes. Define-se $t_i = i \cdot \Delta t$ com $\Delta t = 1$ e $i = 0,1,\ldots,5$.

\subsection*{2.3 Modelo discretizado}
Para cada ponto $t_i$ determinam-se variaveis $x_i$ e $y_i$. A energia aproximada pela regra dos trapezios resulta em:
\[
E \approx \sum_{i=0}^{N-1} \frac{1}{2} \left[ \left( v_i^2 + \sin(t_i) \right) + \left( v_{i+1}^2 + \sin(t_{i+1}) \right) \right] \Delta t,
\]
com
\[
v_i^2 = \left( \frac{x_{i+1} - x_i}{\Delta t} \right)^2 + \left( \frac{y_{i+1} - y_i}{\Delta t} \right)^2.
\]

As condicoes de contorno discretas permanecem $x_0 = 0$, $y_0 = 0$, $x_N = 10$, $y_N = 10$.

\section*{3. Justificativa do Solver IPOPT}
O IPOPT (Interior Point OPTimizer) foi escolhido por:
\begin{itemize}
    \item Ser especializado em problemas nao lineares suaves com restricoes de igualdade e desigualdade.
    \item Utilizar metodos de pontos interiores estaveis e com boa taxa de convergencia.
    \item Integrar-se nativamente ao Pyomo, dispensando codificacao manual de derivadas.
    \item Operar como software livre, com desempenho comparavel a solvers comerciais.
\end{itemize}

\section*{4. Implementacao em Pyomo}
O modelo foi implementado com:
\begin{itemize}
    \item \textbf{ConcreteModel} para representar o problema.
    \item \textbf{RangeSet} para indices de tempo.
    \item \textbf{Var} para coordenadas $x_i$ e $y_i$.
    \item Uma funcao objetivo Python que computa a energia via regra dos trapezios.
    \item \textbf{SolverFactory('ipopt')} apontando para o executavel instalado via Miniconda.
\end{itemize}

\section*{5. Resultados do Caso Base ($N = 5$)}
\subsection*{5.1 Diagnostico do solver}
O IPOPT retornou status \texttt{ok} com condicao de parada \texttt{optimal}. A energia minimizada foi:
\[
E^\ast = 35.949741.
\]

\subsection*{5.2 Trajetoria otima}
A trajetoria obtida e quase linear, conforme a Tabela~\ref{tab:trajetoria}. As coordenadas $x_i$ e $y_i$ sao praticamente iguais, indicando trajeto proximo da diagonal $y = x$.

\begin{table}[H]
    \centering
    \caption{Posicoes otimas no caso base}
    \label{tab:trajetoria}
    \begin{tabular}{ccc}
        \toprule
        Tempo (s) & $x_i$ & $y_i$ \\
        \midrule
        0.0 & 0.000000 & 0.000000 \\
        1.0 & 3.529412 & 3.529412 \\
        2.0 & 5.294118 & 5.294118 \\
        3.0 & 7.058824 & 7.058824 \\
        4.0 & 8.823529 & 8.823529 \\
        5.0 & 10.000000 & 10.000000 \\
        \bottomrule
    \end{tabular}
\end{table}

\subsection*{5.3 Interpretacao}
\begin{itemize}
    \item O termo dominante da energia e $v^2$; $\sin(t)$ introduz pequenas correcoes.
    \item A trajetoria linear uniforme e energeticamente eficiente para ligar os pontos extremos sem outras restricoes.
    \item A convergencia ocorreu sem violacoes, confirmando a adequacao do solver e do modelo.
\end{itemize}

\section*{6. Conclusoes do Caso Base}
\begin{itemize}
    \item A combinacao Pyomo + IPOPT resolve o problema de forma direta e robusta.
    \item O caminho quase retilineo confirma a intuicao fisica de menor distancia com custo quadratico de velocidade.
    \item As condicoes de contorno e relacoes de velocidade foram satisfeitas exatamente.
    \item O modelo serve de base para adicionar novas restricoes ou funcoes objetivo.
\end{itemize}

\section*{7. Experimentos Extras}
Apos validar o caso base foram conduzidas variacoes para analisar sensibilidades e restricoes adicionais.

\subsection*{7.1 Sensibilidade ao refinamento temporal}
O intervalo $[0,5]$ foi subdividido com diferentes numeros de subintervalos $N$. A Tabela~\ref{tab:sensibilidade} mostra a energia obtida em cada caso. Mesmo com energia ligeiramente crescente (devido ao termo $\sin(t)$ ser avaliado em mais pontos positivos), a trajetoria permanece essencialmente linear.

\begin{table}[H]
    \centering
    \caption{Energia em funcao do refinamento temporal}
    \label{tab:sensibilidade}
    \begin{tabular}{cc}
        \toprule
        Subintervalos ($N$) & Energia \\
        \midrule
        5  & 35.9497 \\
        10 & 38.2014 \\
        20 & 39.4223 \\
        40 & 40.0597 \\
        \bottomrule
    \end{tabular}
\end{table}

\subsection*{7.2 Checkpoint intermediario}
Ao impor a passagem pelo ponto $(5,6)$ no instante $t = 2.5$, a energia sobe para 36.364 unidades (acrescimo aproximado de $1{,}2\%$). A trajetoria desvia temporariamente da diagonal para satisfazer a restricao, ilustrando cenarios com inspecoes ou paradas obrigatorias.

\subsection*{7.3 Penalizacao adicional}
Incluiu-se um termo $0.01[(x_i-10)^2 + (y_i-10)^2]$ na funcao objetivo, incentivando aproximacao precoce ao destino. A energia resultante foi 39.390 unidades. As coordenadas intermediarias ficam mais proximas de $(10,10)$, o que pode ser desejavel para manter o robo sob cobertura de comunicacao ou sensores.

\subsection*{7.4 Biblioteca de pistas}
Foram comparadas tres configuracoes padronizadas:
\begin{itemize}
    \item \textbf{Linha direta:} modelo original, energia de 35.950 unidades.
    \item \textbf{Checkpoint central:} obrigatoriedade em $(5,6)$, energia de 36.364 unidades.
    \item \textbf{Corredor guiado:} restricoes $-1.5 \leq y - x \leq 1.5$ e penalizacao leve, energia de 43.080 unidades.
\end{itemize}
Os resultados (energias, status do solver e trajetorias) estao registrados graficamente no notebook, evidenciando o trade-off entre liberdade geometrica e custo energetico.

\subsection*{7.5 Visualizacao animada}
A trajetoria de menor energia entre as pistas comparadas foi animada para facilitar a apresentacao. A animacao destaca a progressao temporal do robo e auxilia em defesas do trabalho ou em comunicacao com clientes.

Esses experimentos mostram que a formulacao e flexivel e suporta demandas adicionais sem perda de robustez numerica.

\end{document}
